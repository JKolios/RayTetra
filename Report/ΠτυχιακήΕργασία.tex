% Εκτύπωση σε χαρτί A4, μία σελίδα ανά φύλλο, με ξεχωριστή σελίδα για τον τίτλο,
% σε γραμματοσειρά 12pt και format άρθρου.
\documentclass[a4paper,oneside,titlepage,12pt]{article}

% Προσοχή στο [cm-default], χωρίς αυτό μπορεί να μην λειτουργούν τα
% μαθηματικά σύμβολα σε ορισμένες εγκαταστάσεις του xelatex!
\usepackage[cm-default]{fontspec}
\usepackage{xunicode}
\usepackage{xltxtra}
\usepackage{cite}

% Γραμματοσειρά
\setmainfont[Mapping=tex-text]{DejaVu Sans}

% Χρήσιμο πακέτο για εισαγωγή εικόνων jpg/png ή άλλων εγγράφων pdf.
\usepackage{graphicx}


\title{Υλοποίηση και σύγκριση αλγορίθμων τομής ευθείας-τετραέδρου στην GPU}
\author{Κολιός Απόστολος - Ιάσων\\cst05023@uop.gr}
\date{}

\begin{document}

\maketitle

\renewcommand{\contentsname}{Περιεχόμενα}
\tableofcontents


\section{Εισαγωγή}

Μια από τις βασικότερες κατηγορίες προβλημάτων στο πεδίο της υπολογιστικής γεωμετρίας είναι τα προβλήματα εντοπισμού τομής (intersection detection) μεταξύ γεωμετρικών σχημάτων (στο επίπεδο) ή στερεών (στον χώρο). Γενικά ορισμένο, το πρόβλημα τομής μεταξύ δύο γεωμετρικών αντικειμένων περιγράφεται ως εξής:
\\
"Με δεδομένα τα γεωμετρικά χαρακτηριστικά δύο αντικειμένων και των συντεταγμένων που ορίζουν την θέση τους να βρεθεί
αν τα αντικείμενα τέμνονται. Αν τέμνονται, να προσδιοριστούν τα γεωμετρικά χαρακτηριστικά της τομής."
\\
Η ύπαρξη αλγοριθμικών λύσεων υψηλής απόδοσης για τα προβλήματα εντοπισμού τομής είναι θεμελιώδους σημασίας για έναν μεγάλο αριθμό εφαρμογών 
σε έναν μεγάλο αριθμό διαφορετικών πεδίων. Παραδείγματα πεδίων στα οποία χρησιμοποιούνται οι λύσεις αυτές είναι , μεταξύ άλλων, τα γραφικά υπολογιστών (και ιδιαίτερα το animation), τα προγράμματα σχεδιασμού υποβοηθούμενου
από υπολογιστή (CAD), τα συστήματα πλοήγησης, τα γεωγραφικά πληροφοριακά συστήματα(GIS) καθώς και ένα ευρύ φάσμα επιστημονικών
και μηχανικών εξομοιωτών.

Σε αυτή την εργασία θα ασχοληθούμε με το πρόβλημα τομής ευθείας-τετραέδρου. Το συγκεκριμένο  πρόβλημα έχει ιδιαίτερη σημασία καθώς τα τελευταία χρόνια η χρήση τεραεδρικών πλεγμάτων (tetrahedral meshes) έχει εφαρμοστεί με επιτυχία για την αναπαράσταση περίπλοκων τρισδιάστατων όγκων σε ένα μεγάλο εύρος εφαρμογών.      

Για το πρόβλημα τομής ευθείας-τετραέδρου έχουν προταθεί διάφοροι αλγόριθμοι με διαφορετικά επίπεδα απόδοσης. Στην εργασία αυτή θα ασχοληθούμε με τον αλγόριθμο που παρουσιάζεται στο ~\cite{PlatisTheoharis03}, καθώς και σε μια ελαφρώς διαφορετική προσέγγιση που προέρχεται από το ~\cite{ericson2005real}.

Σκοπός της εργασίας αυτής είναι η ανάπτυξη μιας εναλλακτικής υλοποίησης των αλγορίθμων αυτών η οποία προορίζεται για εκτέλεση από μονάδες επεξεργασίας γραφικών (GPU). Η χρήση των GPU για την ταχεία εκτέλεση υπολογισμών γενικού σκοπού γνωρίζει ραγδαία αύξηση τα τελευταία χρόνια. Η αρχιτεκτονική των σύγχρονων GPU, η οποία βασίζεται στο μοντέλο παραλληλίας SIMD (Single Instruction Multiple Data), επιτρέπει μια σημαντικότατη αύξηση επιδόσεων σε ορισμένες εφαρμογές σε σύγκριση με αυτές των CPU.        Όπως θα δούμε στην συνέχεια, οι αλγόριθμοι τομής ευθείας-τετραέδρου που χρησιμοποιούμε μπορούν να προσαρμοστούν σε τέτοια μορφή ώστε να εκμεταλλευτούν τα πλεονεκτήματα της αρχιτεκτονικής των GPU και να επιτύχουν αυξημένες επιδόσεις.   



 
\section{Τομή Ευθείας-Τετραέδρου}
\subsection{Ορισμός Προβλήματος}

Το πρόβλημα της τομής ευθείας - τετραέδρου εντάσσεται στην κατηγορία των προβλημάτων τομής πρωτογενών στερεών (primitive solids).
Τα πρωτογενή στερεά

\subsection{Αλγόριθμοι}

Οι αλγόριθμοι Platis και STP, οι διανυσματικοί υπολογισμοί που χρησιμοποιούνται, η ροή ελέγχου στις εκδόσεις 0,1,2

\section{Υλοποίηση σε GPU}
\subsection{Το μοντέλο επεξεργασίας GPGPU}

Γενικά το μοντέλο επεξεργασίας GPGPU, το OpenCL και γιατί το χρησιμοποιούμε στην παρούσα εφαρμογή

\subsection{Στοιχεία Υλοποίησης}

Η δομή του κώδικα, πώς κατανέμεται η εργασία στην GPU, πώς οργανώνεται η μνήμη στο σύστημα και στην GPU, Software που χρησιμοποιήθηκε,ίσως και οι διαφορές λειτουργίας ανάλογα με το είδος της GPU

\subsection{Χρήση Προγράμματος}

Απαιτήσεις συστήματος,μορφή αρχείων εισόδου-εξόδου,παράμετροι(arguments)

\section{Απόδοση}
\subsection{Μεθοδολογία Μέτρησης Απόδοσης}

Μεθοδολογία Μέτρησης Απόδοσης

\subsection{Αποτελέσματα Μετρήσεων}

αποτελέσματα Benchmark

\subsection{Σύγκριση Απόδοσης}

συγκρίσεις με CPU,A

\section{Συμπεράσματα}

\bibliography{citations}{}
\bibliographystyle{alpha}
\end{document}

\end{document}